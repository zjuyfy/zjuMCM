\documentclass[12pt, a4paper, oneside]{ctexart}
\usepackage{graphicx,amsmath,amssymb,geometry}
\usepackage{enumerate,enumitem}
\usepackage{autobreak}
\usepackage{breqn}
\geometry{left=2.5cm, right=2.5cm, top=2.5cm, bottom=2.5cm}

\usepackage{booktabs}

\usepackage{fancyhdr}
\renewcommand{\footrulewidth}{0.4pt}
\renewcommand{\headrulewidth}{0.4pt}
\pagestyle{fancy}
\fancyhf{}
\cfoot{\thepage}


\newtheorem{assumption}{假设}[section]

\CTEXsetup[name={,、\hspace{-1em}},number={\chinese{section}}]{section}






\begin{document}
    \begin{abstract}
        
    \end{abstract}
\newpage
    \section{问题重述}
    \subsection{问题背景}
    如今,化石燃料和环境保护已成为世界上最热⻔的问题[1]。世界范围内的主要交 通工具以石油为主,造成了严重的环境污染。由于化石燃料的资源有限,而且总是导致污染,因此向更清洁的能源过渡是不可避免的。为了实现这一目标,以新能源汽⻋ 为代表的电动汽⻋被公认为21世纪汽⻋产业转型的主要发展方向[2]。

    众所周知,电动汽⻋效率高、噪音低、污染几乎为零[3]。电动汽⻋的优势显而易 ⻅。因此,它们已在世界各地广泛使用。推动电动汽⻋的规模化发展,需要完善相应的基础设施。充电站作为电动汽⻋设施建设的重要组成部分,对整个电动汽⻋产业的 发展至关重要。我们需要在哪里和多少个充电站建设?选择正确的位置和估计充电站 的数量非常重要。

    随着温室效应和空气污染问题的加剧,各国都在寻找新能源来替代传统燃料,如原油或柴油,以缓解日益严重的空气问题。自混合动力汽⻋和燃气汽⻋问世以来,新型清洁汽⻋的探索仍在不断进行。目前,以特斯拉为首的电动汽⻋将在更大程度上突破能源和经济的限制,更好地平衡快速增⻓的汽⻋需求与环境的关系。充电站数量合适,距离合适,对于电动汽⻋的普及至关重要。与加油站相比,电动汽⻋充电站占用的空间更小,具有更高的安全系数,可以更好地分布在街道和社区,让人们更方便、更高效地使用。然而,电动汽⻋的推广并非一蹴而就。要逐步扩大电动汽⻋覆盖面,不断完善电动汽⻋充电站网络,最终完成汽油和柴油汽⻋的终结运营。另外,不同国 家的经济、文化条件不同,需要根据自己的具体情况确定推广时间和推广范围,才能取得更好的效果。要逐步扩大电动汽⻋覆盖面,不断完善电动汽⻋充电站网络,最终完成汽油和柴油汽⻋的。
    \subsection{具体问题重述}
    \subsubsection{第一问}
    开发一算法来确定某地区的充电站系统,包括充电站的位置和数量,并预测未来增长情况。

    \subsubsection{第二问}
    根据开发的算法进行结果的分析和讨论,找出未来影响电动汽车发展战略的主要因素,预测电动汽车未来发展的趋势。

    \section{问题分析}
    
    \section{符号说明}
    \begin{table}[h]
        \caption{符号说明}
        \centering
        \renewcommand{\arraystretch}{2}
        \begin{tabular*}{\textwidth}{c||l}%l左对齐,c居中,r右对齐,p指定列宽度
          \toprule[1.5mm]
          \Large{\textbf{符号}} & \Large{\textbf{说明}} \\
          \midrule[1.5pt]
          \(d_{ij}\) & OD距离成本矩阵,表示i地到j地的距离 \\\hline
          \(q_i\)&需求列表,表示i地的车辆数量(即对充电桩的需求量)\\\hline
          \(p\)&总充电桩数量\\\hline
          \(n_i\)&供应列表,表示i地安装的慢充充电桩数量\\\hline
          \(\rho_{ij}\)&分配矩阵,表示i地的需求以\(\rho_{ij}\)的比例分配给j地的慢充充电桩\\\hline
          \(c_j\)&容量列表,表示i地能够供应的容量,等于i地建设的慢充充电桩数量\\\hline
          &\\\hline
          \bottomrule[1mm]
        \end{tabular*}
      \end{table}
    \section{基本假设}
    \begin{assumption}\label{asmp:lcpark}
        每个目的地的停车场数量足以供应目的地的车流量(无需在目的地建造新的停车场)
    \end{assumption}
    \begin{assumption}\label{asmp:lcload}
        慢充不会对电网产生高峰期的负荷,只需算总耗电。原因是慢充总可以智能选择一天当中电网负荷低的时候充电。
    \end{assumption}
    \begin{assumption}\label{asmp:lcoccupy}
        只有电量低于10\%的电动车会占用慢充车位充电并且错开(即不会所有车同时电量低于10\%),所以慢充不考虑排队只考虑总耗电;同时每天分为白天和晚上两个充电时期。
    \end{assumption}
    

    \section{模型建立}
    \subsection{充电桩选址模型}
    \subsubsection{充电桩分类}
        我们将充电站分为两种,分别是短时停靠充电站(简称快充,sc)和长时停靠充点站(简称慢充,lc)

        长时停靠充电站用于在人们长时间停车(平均超过5h,用普通功率,车可以充满),

        短时停靠充电站用于在人们长途旅行时临时充电(起到类似于加油站的作用,需在半小时内充满一辆汽车)
    \subsubsection{约束条件以及概念}
    \begin{enumerate}[label = \alph*)]
        \item \textbf{成本}\\
            快充成本为地价和充电桩费用的叠加
            \begin{align}\begin{autobreak}
	            f_{sc}=\sum\limits_{i}(f_{ci}+f_{li})
            \end{autobreak}\end{align}

        

            慢充建在停车场,价格远低于快充且单价固定,所以对成本的限制转化为对数量的限制\begin{align}\begin{autobreak}
	f_{lc}\propto p
\end{autobreak}\end{align}
        \item \textbf{区域总耗电量q}\\
            汽车的日均总耗电量q可由如下公式计算得到:
            \begin{align}\begin{autobreak}
	q=\frac{L}{365}\cdot \frac{F_k}{100}\cdot Po\cdot V_p\cdot \alpha_e
\end{autobreak}\end{align}
            其中L为年平均行驶里程,\(F_k\)为百公里耗电,\(Po\)为地区人口,\(V_p\)为地区人均私家车数量,\(\alpha_e\)为电动车占比,可由电动车发展模型得到。以上数据均可以从互联网中查得。
        \item \textbf{OD距离矩阵d\(_{ij}\)}\\
            距离矩阵可以通过道路实际位置和道路拓扑结构通过对每两个节点求最短路径算出。道路数据从互联网上查得。
        \item \textbf{电网负荷L\(_a\)}\\
            由于中国电网很强大,且现在电动车数量稀少,我们初步模型暂时不考虑这个条件作为约束,但可以根据结果计算对电网造成的负荷。计算方法如下:\\
            慢充考虑一整天的负荷\(L_{lc}=\beta q/24\),\(\beta\)为损耗系数。除以24得到平均每小时负荷\\
            快充考虑高峰时期不间断充电造成的负荷:\(L_{sc}=P_{sc}=F_k\cdot E_k/t_{sc}\),其中\(E_k\)为续航里程,\(t_{sc}\)为快充充满所需时间。以上数据均可以从互联网中查得。

        \item \textbf{单位承载力c\(_0\)}\\
            单个充电桩所能够承载几辆车充电。我们默认\(c_0=1\)。
        \item \textbf{单次充电跨越时间T}\\
            依据假设\ref{asmp:lcoccupy},电动车从满电到10\%电量跨越的半天数\begin{align}\begin{autobreak}
	T=\left\lfloor 0.9\frac{E_k}{L/(2\cdot 365)}\right\rfloor 
\end{autobreak}\end{align}
        \item \textbf{各节点充电桩数量n\(_i\)}\\
            各个节点充电桩数量总和应当等于充电桩总数:\(\sum\limits_i n_i=p \)
        \item \textbf{各节点需求q\(_i\)}\\
            每个节点的需求即每个节点在每个半天的所有需要充电(即电量低于10\%)的车的数量\begin{align}\begin{autobreak}
	q_i=Po\cdot V_p\cdot\alpha_e/T
\end{autobreak}\end{align}
        \item \textbf{分配矩阵\(\mathbf\rho_{ij}\)}\\
            代表i地的需求中\(\rho_{ij}\)的比例由j地满足,因此有\(\sum\limits_j \rho_{ij}=1\)
        \item 
    \end{enumerate}
    \subsubsection{单目标优化}
    \(d_{ij},\,q_i,\,p\)作为已知量,以\(n_j,\,\rho_{ij}\)为变量进行优化,综合上面的分析可得优化方程为:
    \begin{align}
        &\min  f(\rho)=\sum_i \sum_jd_{ij} q_j \rho_{ij}\\
        &\begin{cases}
            \sum_i n_i=p &\text{(共有p个充电桩)}\\
            \sum_j \rho_{ij}=1&\text{(比例系数和为1)}\\
            \sum_i q_{i}\rho_{ij}\leqslant c_j=n_j \cdot c_0&\text{(承载力限制)}
        \end{cases}
    \end{align}

    \subsection{电动车增长模型}
    \subsubsection{相关性分析}
    通过对于以往研究电动汽车发展的论文,我们注意到影响我国电动汽车保有量变化的因素有多种,如经济因素,包括人均GDP、居民收入、经济产业结构、居民消费水平等;社会因素,包括城市人口、城市化率、失业率、拥塞成本等;环境因素,包括公路网规模、基础设施完善度等。考虑长期完整数据的可获得性,我们在这里初步选取了城市人口数、城市化水平、人均GDP、居民消费水平、公路里程、第一产业占比、第二产业占比、第三产业占比这8个代表性的影响因素进行分析。

    首先我们将以上八个因素与电动汽车保有量进行了Spearman相关系数相关性分析,得到如下相关系数热力图。(八个因素的数据来自于《中国统计年鉴(2021)》\cite{cite:统计年鉴},电动汽车保有量来自于《2022年中国新能源汽车行业市场前景及投资研究预测报告》\cite{cite:预测报告})
    \subsubsection{主成分分析}
    主成分分析法是运用“降维”思想,把多个指标变换成少数综合指标的多元统计方法,这里的综合指标就是主成分。每个主成分都是原始变量的线性组合,彼此相互独立,并保留了原始变量绝大部分信息。其本质是通过原始变量的相关性,寻求相关变量的综合替代对象,并且保证了转化过程中的信息损失最小 。

    我们对城市人口数、城市化水平、人均GDP、居民消费水平、公路里程、第三产业占比六个因素通过SPSSPRO进行主成分分析,来综合评价每年度的社会经济发展水平。
    
    \subsection{充电桩增长模型}
    充电桩基本只和电动车增长有关,所以我们依据电动车的增长模型所算得的数据再代入选址模型可以得到对充电桩选址的动态预测模型。
    \section{模型求解}
    \subsection{问题一的求解}
    \subsubsection{充电桩选址}
    我们选用进化算法求解得到结果d
    \subsection{问题二的求解}
    \subsubsection{相关性分析}
    首先我们将以上八个因素与电动汽车保有量进行了Spearman相关系数相关性分析,得到如下相关系数热力图(图\ref{fig:相关系数热力图})。(八个因素的数据来自于《中国统计年鉴(2021)》\cite{cite:统计年鉴},电动汽车保有量来自于《2022年中国新能源汽车行业市场前景及投资研究预测报告》\cite{cite:预测报告})
    \begin{figure}[h]
        \centering
        \includegraphics[width=0.8\textwidth]{pic/相关系数热力图.png}
        \caption{相关系数热力图}
        \label{fig:相关系数热力图}
    \end{figure}

    注意到第一产业占比、第二产业占比与电动汽车保有量相关性较低,所以我们在后续的主成分分析时略去这两种因素。
    \subsubsection{主成分分析}
    我们得到KMO值=0.58,显著值p<0.001,水平上呈现显著性,拒绝原假设,各变量间具有相关性,主成分分析有效,效度虽然在0.6以下,但尚能接受。
    \begin{figure}[h]
        \centering
        \includegraphics[width=0.8\textwidth]{pic/KMO.png}
        \label{fig:方差解释表}
    \end{figure}
    \begin{figure}[h]
        \centering
        \includegraphics[width=0.8\textwidth]{pic/特征根.png}
        \caption{特征根}
        \label{fig:特征根}
    \end{figure}

    上面方差解释表中,我们发现第一主成分的特征值λ=5.657>1,,且远大于其他特征值, 累积方差贡献率为94.286\%, 即可以解释总体数据的94.286\%。由碎石图可知,在第二个主成分开始特征值下降坡度变缓,且前两个主成分累积解释贡献率超过90\%,因此我们选择两个主成分进行分析。

    从下面的因子载荷矩阵热力图中可以分析每个主成分中隐变量的重要性,其中第一个主成分与前五个因素都具有较大的相关性(0.9以上),我们概括为“城市总体发展程度”。主成分2与第三产业占比的相关程度较大,可以概括为“服务业发展程度”。
    

    根据表\ref{fig:载荷系数}我们得到模型的公式:
    \begin{align}\begin{autobreak}
        F1=0.176\times \mbox{城市人口数}(P1) 
        + 0.173\times \mbox{城市化水平}(P2) + 0.174\times \mbox{人均GDP} (P3) 
        + 0.175\times \mbox{居民消费水平}(P4) 
        +0.174\times \mbox{公路里程}(P5) 
        +0.157\times \mbox{第三产业占比}(P8)
    \end{autobreak}\end{align}
    \begin{align}\begin{autobreak}
	    F2=0.299\times \mbox{城市人口数}(P1)
        -0.549\times \mbox{城市化水平}(P2)
        -0.511\times \mbox{人均}GDP (P3)
        -0.289\times \mbox{居民消费水平}(P4)
        -0.433\times \mbox{公路里程}(P5)
        +1.635\times \mbox{第三产业占比}(P8)
    \end{autobreak}\end{align}
    
    由上可以得到: 
    \begin{align}\begin{autobreak}
        F=(0.943/0.989)\times F1+(0.046/0.989)\times F2
    \end{autobreak}\end{align}

    \begin{figure}[h]
        \centering
        \includegraphics[width=0.8\textwidth]{pic/因子载荷表.png}
        \label{fig:载荷系数}
    \end{figure}

    由图\ref{fig:第一主成分值}可知:第一主成分值与年份呈显著的线性正相关, 故利用线性回归模型得出第一主成分与年份的关系, 如式:
    \begin{dmath}
        y=0.4071x-820.4
    \end{dmath}
    \begin{figure}[h]
        \centering
        \includegraphics[width=0.8\textwidth]{pic/第一主成分值.jpg}
        \caption{第一主成分值与年份呈显著的线性正相关}
        \label{fig:第一主成分值}
    \end{figure}


    \section{模型评价}

    \section{参考文献}
    \begin{thebibliography}{99}
        \bibitem{cite:统计年鉴}中华人民共和国统计局.中国统计年鉴(2021)[M].北京:中国统计出版社,2022.
        \bibitem{cite:预测报告}
    \end{thebibliography}
    \section{附录}
\end{document}
