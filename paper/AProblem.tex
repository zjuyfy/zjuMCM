\documentclass[12pt, a4paper, oneside]{ctexart}
\usepackage{graphicx,amsmath,amssymb,geometry}
\usepackage{enumerate,enumitem}
\geometry{left=2.5cm, right=2.5cm, top=2.5cm, bottom=2.5cm}


\usepackage{fancyhdr}
\renewcommand{\footrulewidth}{0.4pt}
\renewcommand{\headrulewidth}{0.4pt}
\pagestyle{fancy}
\fancyhf{}
\cfoot{\thepage}







\begin{document}
    \begin{abstract}
        
    \end{abstract}
\newpage
    \section{问题重述}
    \subsection{问题背景}
    如今,化石燃料和环境保护已成为世界上最热⻔的问题[1]。世界范围内的主要交 通工具以石油为主,造成了严重的环境污染。由于化石燃料的资源有限,而且总是导致污染,因此向更清洁的能源过渡是不可避免的。为了实现这一目标,以新能源汽⻋ 为代表的电动汽⻋被公认为21世纪汽⻋产业转型的主要发展方向[2]。

    众所周知,电动汽⻋效率高、噪音低、污染几乎为零[3]。电动汽⻋的优势显而易 ⻅。因此,它们已在世界各地广泛使用。推动电动汽⻋的规模化发展,需要完善相应的基础设施。充电站作为电动汽⻋设施建设的重要组成部分,对整个电动汽⻋产业的 发展至关重要。我们需要在哪里和多少个充电站建设?选择正确的位置和估计充电站 的数量非常重要。

    随着温室效应和空气污染问题的加剧,各国都在寻找新能源来替代传统燃料,如原油或柴油,以缓解日益严重的空气问题。自混合动力汽⻋和燃气汽⻋问世以来,新型清洁汽⻋的探索仍在不断进行。目前,以特斯拉为首的电动汽⻋将在更大程度上突破能源和经济的限制,更好地平衡快速增⻓的汽⻋需求与环境的关系。充电站数量合适,距离合适,对于电动汽⻋的普及至关重要。与加油站相比,电动汽⻋充电站占用的空间更小,具有更高的安全系数,可以更好地分布在街道和社区,让人们更方便、更高效地使用。然而,电动汽⻋的推广并非一蹴而就。要逐步扩大电动汽⻋覆盖面,不断完善电动汽⻋充电站网络,最终完成汽油和柴油汽⻋的终结运营。另外,不同国 家的经济、文化条件不同,需要根据自己的具体情况确定推广时间和推广范围,才能取得更好的效果。要逐步扩大电动汽⻋覆盖面,不断完善电动汽⻋充电站网络,最终完成汽油和柴油汽⻋的。
    \subsection{具体问题重述}
    \subsubsection{第一问}
    开发一算法来确定某地区的充电站系统,包括充电站的位置和数量,并预测未来增长情况。

    \subsubsection{第二问}
    根据开发的算法进行结果的分析和讨论,找出未来影响电动汽车发展战略的主要因素,预测电动汽车未来发展的趋势。

    \section{问题分析}
    
    \section{符号说明}

    \section{基本假设}
    \begin{enumerate}
        \item 每个目的地的停车场数量足以供应目的地的车流量(无需在目的地建造新的停车场)
        \item 慢充不会对电网产生高峰期的负荷,只需算总耗电
        \item 只有电量低于10\%的电动车会占用慢充车位充电并且错开,所以慢充不考虑排队只考虑总耗电
    \end{enumerate}

    \section{模型建立}
    \subsection{充电桩选址模型}

    \subsection{电动车增长模型}

    \subsection{充电桩增长模型}

    \section{模型求解}

    \section{模型评价}

    \section{参考文献}

    \section{附录}
\end{document}
